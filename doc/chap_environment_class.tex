\chapter{The Environment Class}
\label{chap:environment_class}

The Environment Class (smarts/env.py) is the interface that tests will use to
load compilers, mpi installations and libraries that are specified in a
Environment.yaml file. 

SMARTS uses the Environment Class internally to load and store a parsed
Envrionment.yaml file, but has public methods that are to be used by tests to
load different Modsets.

Each \testRun method of a test will be passed an instance of the Environment
class that has been loaded with the Environment.yaml that was specified by the
{\tt smarts.py} command line. Each test can use this instance of the
Environment to load modsets as they choose.

The Environment class contains two public routines which tests can use to load
modsets: \listModsets and \loadModset, which are described in the two sections
below.

\section{List Modsets}
\label{sec:list-modsets}

\begin{lstlisting}[language=Python, 
                   caption={list\_modset Definition},
                   label=list_modset_def]
def list_modsets(self, name=None, *args, **kwargs):
    """ Return a list of modsets found in the parsed environment.yaml file, if
    name is specified then modset names that contain that name will be
    returned.

    Keyword arguments:
    name -- Name to specify specific modset(s) name (String)
    """
\end{lstlisting}

The \listModsets command can be used retrive the names of modsets in a list
which can later be used in the \loadModset command (see
\ref{sec:load-modsets}). As seen in \ref{list-modset-def}, if \listModsets is
ran and {\tt name=None} then all the Modsets contained will be returned;
however, if {\tt name} is set, the Modsets that contain the name specified in
name will be retrived.

For instance, if SMARTS was initiated with a Environment.yaml file that
contained the modsets:
{\tt
\begin{itemize}
\item GNU-9.1.0
\item GNU-8.3.0
\item INTEL-19.0.2
\end{itemize}
}

The following calls to {\tt Environment.list\_modset} would return the
following:

{\tt
\begin{itemize}
\item env.list\_modset() - ['GNU-9.1.0', 'GNU-8.3.0', 'INTEL-19.0.2']
\item env.list\_modset(name='GNU-9.1.0') - ['GNU-9.1.0']
\item env.list\_modset(name='GNU') - ['GNU-9.1.0', 'GNU-8.3.0']
\end{itemize}
}


\section{Load Modsets}
\label{sec:load-modsets}

\begin{lstlisting}[language=Python, 
                   caption={load\_modset Definition},
                   label=load_modset_def]
def load_modset(self, modsetName, *args, **kawrgs):
    """ Completely load the modset, modsetName to be used by a single test.
    This function completely loads a modset (compiler, mpi implementation, and
    all libraries).

    To load a compiler, this function will alter the PATH environment variable
    for the current process (single test) and prepend the compiler path to it.
    If the compiler is specified as a module, it will be loaded via the
    lmod Python interface (`module python load ...`)

    MPI implementation will be loaded in a similar manner to compilers.

    Both MPI and Compilers will be checked to ensure that the correct version
    is installed by running the compiler executables specified in the
    executables section of the compiler or MPI env.yaml sections with
    `--version` and checking the versions in the env.yaml file match correctly.

    Libraries will be loaded by creating ENV_NAME as an environment variable
    and assigning to it the value specified in value.

    modsetName -- Name of the modset to be loaded (String)
\end{lstlisting}

From a modset name, returned from \listModsets or entered in manual, can be
used in \loadModset to load a specific compiler i.e.:

\begin{lstlisting}[language=Python]
gnu_modsets = env.list_modsets(name="GNU-9.1.0")
env.load_modset(gnu_modsets[0])
\end{lstlisting}

When \loadModset is called on a modset, it will load the \compiler, \mpi and
\libs section in the following order and in the following way:

\begin{enumerate}

\item Compiler 

Depending on how the compiler is specified in the Environment.yaml file, as a
module or as an installation path, it will be loaded either by using the lmod
{\tt module load} command or by prepending the installation's binary ({\tt
bin}) directory to the proccess \pathenv environment variable. 

The lmod program has the ability to generate Python commands which can be used
to alter the program's environment accordingly if executed. SMARTS will take
these commands generated by LMOD and executed them. See ({\tt
Environment.\_lmod\_load()} for more details.

As stated above, if the compiler is listed as a compiler installation path,
then '{\tt \\bin}' will be appended to the installation path and that path will
be appended to the \pathenv environment variable.

In both cases, loading the compiler in such a way will allow calls to Python's
{\tt os.system()} to use the loaded compiler.

After the above steps, SMARTS will check to see if the correct compiler version
has been loaded. This is done by running all of the executables listed in the
\executables section with {\tt --version} and checking to see if the version
specified in \version is in the output.


\item MPI

If specified, the MPI section will be loaded in the same way as the compiler
section above.

\item Libs 

Libraries in the \libs section of the Environment.yaml file will be loaded in
the order that they are listed. Depending on how the library is specified, as
either a module (\module) or as an environment variable value pair (\name,
\value) the library will be loaded using the lmod command or by creating an
environment variable and setting it equal to value.

As with the compiler and MPI sections above, SMARTS will use the lmod python
tool to generate commands and then execute the commands to update its current
runtime environment. 

For environment variable value pairs, it will create a new environment variable
with \name (or overwrite one if the name is already present) and set it to
\value.

As opposed to how the \compiler and \mpi sections are loaded, libraries are not
tested that they are loaded correctly.

\end{enumerate}
