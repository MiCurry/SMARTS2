\documentclass[11pt]{report}

\usepackage[dvipsnames]{xcolor}
\usepackage{listings}
\usepackage{xspace}
\usepackage{hyperref}
\usepackage{color}

\setlength{\topmargin}{0in}
\setlength{\headheight}{0in}
\setlength{\headsep}{0in}
\setlength{\textheight}{9.0in}
\setlength{\textwidth}{6.5in}
\setlength{\evensidemargin}{0in}
\setlength{\oddsidemargin}{0in}

%%%%%%%%%%%%%%%%%%%%%%%%%%%%%%%%%%%%%%%%%%%%%%%%%%%%%%%%%%
% SMARTS' Keywords
%%%%%%%%%%%%%%%%%%%%%%%%%%%%%%%%%%%%%%%%%%%%%%%%%%%%%%%%%%

% Environment.yaml Keywords
\newcommand{\compiler}{{\tt Compilers}\xspace}
\newcommand{\modset}{{\tt Modsets}\xspace}
\newcommand{\libs}{{\tt Libs}\xspace}
\newcommand{\mpi}{{\tt MPI}\xspace}
\newcommand{\module}{{\tt Module}\xspace}
\newcommand{\version}{{\tt Version}\xspace}
\newcommand{\pathname}{{\tt Path}\xspace}
\newcommand{\executables}{{\tt Executables}\xspace}
\newcommand{\name}{{\tt Name}\xspace}
\newcommand{\valueenv}{{\tt Value}\xspace}
\newcommand{\pathenv}{{\tt PATH}\xspace}
\newcommand{\run}{{\tt run}\xspace}
\newcommand{\hpcoptions}{{\tt HPC\_Options}\xspace}

% Environment.class Keywords
\newcommand{\listModsets}{{\tt list\_modsets}\xspace}
\newcommand{\loadModset}{{\tt load\_modset}\xspace}

% Test Keywords
\newcommand{\testRun}{{\tt run}\xspace}
\newcommand{\testName}{{\tt test\_name}\xspace}
\newcommand{\testDescription}{{\tt test\_description}\xspace}
\newcommand{\npcus}{{\tt nCPUs}\xspace}
\newcommand{\dependencies}{{\tt dependencies}\xspace}
\newcommand{\launchjob}{{\tt HPC.launch\_job}\xspace}
\newcommand{\launchscript}{{\tt HPC.launch\_script}\xspace}


%%%%%%%%%%%%%%%%%%%%%%%%%%%%%%%%%%%%%%%%%%%%%%%%%%%%%%%%%%
% Yaml Listing Definition
%%%%%%%%%%%%%%%%%%%%%%%%%%%%%%%%%%%%%%%%%%%%%%%%%%%%%%%%%%
\newcommand\YAMLcolonstyle{\color{red}\mdseries}
\newcommand\YAMLkeystyle{\color{black}\bfseries}
\newcommand\YAMLvaluestyle{\color{blue}\mdseries}

\makeatletter

% here is a macro expanding to the name of the language
% (handy if you decide to change it further down the road)
\newcommand\language@yaml{yaml}

\expandafter\expandafter\expandafter\lstdefinelanguage
\expandafter{\language@yaml}
{
keywords={true,false,null,y,n},
keywordstyle=\color{darkgray}\bfseries,
basicstyle=\YAMLkeystyle,                   % assuming a key comes first
sensitive=false,
comment=[l]{\#},
morecomment=[s]{/*}{*/},
commentstyle=\color{purple}\ttfamily,
stringstyle=\YAMLvaluestyle\ttfamily,
moredelim=[l][\color{orange}]{\&},
moredelim=[l][\color{magenta}]{*},
moredelim=**[il][\YAMLcolonstyle{:}\YAMLvaluestyle]{:}, % switch to value style at :
morestring=[b]',
morestring=[b]",
literate =    {---}{{\ProcessThreeDashes}}3
              {>}{{\textcolor{red}\textgreater}}1
              {|}{{\textcolor{red}\textbar}}1
              {\ -\ }{{\mdseries\ -\ }}3,
}

% switch to key style at EOL
\lst@AddToHook{EveryLine}{\ifx\lst@language\language@yaml\YAMLkeystyle\fi}
\makeatother

\newcommand\ProcessThreeDashes{\llap{\color{cyan}\mdseries-{-}-}}
%%%%%%%%%%%%%%%%%%%%%%%%%%%%%%%%%%%%%%%%%%%%%%%%%%%%%%%%%%

\lstset{basicstyle=\footnotesize\ttfamily,
        breakatwhitespace=false,
        breaklines=true,
        numbers=left,
        numberstyle=\tiny,
        frame=single,
        float
}

\hypersetup{
    colorlinks=true,
    linkcolor=blue
}

\title{SMARTS - Simple MPAS Atmosphere Regression Testing System - User Guide}
\date{v.000, 21 January 2020}

\begin{document}


\begin{titlepage}
\maketitle
\end{titlepage}


\tableofcontents


\chapter{Introduction}
\label{chap:intro}

This document is a placeholder for a future user document for SMARTS.

\begin{lstlisting}[language=yaml,
                   caption={YAML Example Code Listing},
                   label=yaml_exmaple]
// example.yaml Description:
    Name: Cheyenne
    Max Cores: 4
    Modules: True
    LMOD_CMD: /path/to/lmod_cmd
    HPC: PBS

Modsets:
    Libs:
        - One
        - Two
        - Three
\end{lstlisting}

\begin{lstlisting}[language=Python,
                   caption={Example Python Listing},
                   label=python_example]
// example_test.py
import time

class example_test:
    test_name = "Example Test"
    test_description = "This test makes a file and checks to see if it was" \
                       "created or not"
    nCPUs = 1
    dependencies = [None]

    def run(self, env, result, src_dir, test_dir, hpc):
        pass
\end{lstlisting}

\chapter{Command Line Interface - smarts.py}
\label{chap:smarts_commandline}

SMARTS can currently only be launched from the command line using Python via
\smartspy. In the future there may be other options for launching SMARTS.

The \smartspy command line program is a Python3 program and should be launched
by using Python3. The \smartspy interface is built in the similar manner to the
Git commandline tool in that there are commands and sub-commands. Different to
the Git commandline tool, \smartspy has a number of options that are required
to run.

Listing \ref{lst:smarts_help} shows the usage and help message for \smartspy.
There are three required arguments, which are listed in the 'Required
Arguments' section. These three requirements are:

\begin{itemize}
    \item {\tt -e/--env-file env.yaml} - The Envrionment.yaml file that
    describes the current machine. This file describes the machine and any
    specified compilers or libraries to SMARTS so that tests are able to load
    and unload tests across different machines. See Chapters
    \ref{chap:environment_file_spec} and \ref{chap:environment_class} for more
    information on specifying and using the environment.yaml file.

    \item {\tt -s/--src-dir dir} - The directory of changes to test. This
    directory path will be passed to each test (as the {\tt src\_dir} argument)
    and will allow tests to copy soruce code or executables for regression
    testing. This directory can be located anywhere on the filesystem. See
    Chapter \ref{chap:tests} for more infromation on using the directory from
    the {\tt -s} argument in tests.

    \item {\tt -t/--test-dir dir} - The directory that contains the desired
    tests to run. This argument does two things. One, it shows SMARTS where
    tests can be loaded. SMARTS will look in this directory for valid tests and
    will only load tests from this directory. Secondly, it passes this path to
    each test which allows tests to have access to any supplementary files that
    they may need. For instance this could be a namelist or streams file that
    is specific for a spcific test.
\end{itemize}

All three of these requirments arguments must be specified or \smartspy will
return an error. After these required arguments commands and their subcommands
may be specified. The two commands and their subcommands are:

\begin{itemize}
    \item {\tt list } - The \listcmd can be used to display infromation of the
    current SMARTS system. At present, the list command can only be used to
    list tests and test information, but in the future it may be expanded
    to print other infromation. The subcommands for \listcmd are:
    \begin{itemize}
        \item {\tt list tests} - List the valid and invalid tests that are
        found the the directory passed to the {\tt -t/--test-dir}. See
        \ref{sec:list} for more information.
        \item {\tt list test test-name[s]} - Print out the additional
        infromation on a specific test or test(s). See \ref{sec:list} for
        additional information.
    \end{itemize}
    \item {\tt run test-name[s]} - Run the specified tests. See \ref{sec:run}
    for more information.
\end{itemize}

\begin{lstlisting}[language=Clean,
                   caption={smarts.py Help Message},
                   label=lst:smarts_help]
usage: smarts [-h] [-e env.yaml] [-s dir] [-t dir] [-v level] {list,run} ...

A regression testing system for MPAS

optional arguments:
  -h, --help            show this help message and exit

Required arguments:
  -e env.yaml, --env-file env.yaml
                        The location of the env.yaml file
  -s dir, --src-dir dir
                        The directory that holds the code to test changes
                        (MPAS-Model)
  -t dir, --test-dir dir
                        The location of the test directory

Optional arguments:
  -v level, --verbose level
                        Output debug level

subcommands:
  command description

  {list,run}            Sub-command help message
    list                List SMART's tests, test suites and compilers
    run                 Run a test or a test-suite by name
\end{lstlisting}

\section{List}
\label{sec:list}

An example of the \listtest and its subcommands can be seen in Listing
\ref{lst:smarts_list_tests_example}. The \listtest command will display two
groups of tests. One for valid tests and another for invalid tests. Valid tests
are tests that contain all the necessary parts to be successfully loaded and
launched by SMARTS. Invalid tests, on the other hand, are tests that SMARTS
could not load for one reason or another. Tests that contain syntax errors or
are missing critical parts of a SMARTS test will be placed in invalid tests,
along with the reason they were invalid.

\begin{lstlisting}[language=Clean,
                   caption={smarts.py List Tests Example},
                   label=lst:smarts_list_tests_example]
>>> python3.py smarts.py -e ./envs/cheyenne.yaml -s ~/mpas_model_changes -t
./mpas_tests list tests
Tests found in: /users/mcurry/smarts/smarts_mpas_test:
Valid tests:
  - bit_for_bit_tests -- Bit for Bit Test
  - compile_test -- Test that MPAS compiles
  - mpas_reg_tests -- Complete MPAS Regression Test
  - smoke_test -- Smoke Test
  - sst_update_test -- SST Update Test

Invalid tests: (These tests were not able to be loaded)
  x restart_test.restart_test -- EOL while scanning string literal (restart_test.py, line 12)
\end{lstlisting}

To find out more information on a test, one can run the {\tt list test
[test-name]} command and smarts.py will print out additional information test
information. This command can be used with any number of test names. Listing
\ref{lst:smarts_list_test_name_example} provides an example for this command.

\begin{lstlisting}[language=Clean,
                   caption={smarts.py List Test Info},
                   label={lst:smarts_list_test_name_example}]
>>> python3.py smarts.py -e ./envs/cheyenne.yaml -s ~/mpas_model_changes -t
./mpas_tests list test mpas_reg_tests
    Run name: mpas_reg_tests
   Long name: Complete MPAS Regression Test
 Description: Run all MPAS regression tests
       ncpus: 1
Dependencies: ['compile_test', 'smoke_test', 'bit_for_bit_tests', 
               'sst_update_test']
\end{lstlisting}

\section{Run}
\label{sec:run}

The \runcmd can be used with one or more test names. Doing so, will launch the
specified tests through SMARTS, if those tests are valid. If they are not, an
error will be reported and no tests will be ran.

If all tests are valid and able to be loaded correctly, then SMARTS will create
a new run directory will be created. Within this directory tests will be given
their own directory, which will become their current working directory.
Grabbing the results of one test (say a compiled executable from a compile
test) can be done by using the current working directory path. All main test run
directories will named {\tt run-smarts-YYYY-MM-DD-hh.mm.ss} with the date and
time pieces being the date and time whne SMARTS was launched. 

If a test has a test (or multiple tests) listed in its dependencies attribute,
and that test is not specified in the run command, then it will automatcially
be loaded and ran. Tests that are dependencies for other tests will be ran
before their dependents and dependents will not run if any of their
dependencies fail.

For instance, in \ref{lst:smarts_list_test_name_example} specifying
mpas\_reg\_testing to run will load and run all of the dependencies listed in
the Dependencie's list. If any of these tests fail, then the result of
mpas\_reg\_testing will be marked as {\tt INCOMPLETE}.

Tests will only be loaded once per \smartspy command. It is not possible for
SMARTS to run two of the same tests twice. This includes if a test is specified
as a dependency and is specified to run via \smartspy.

Running the same test twice should be done with seperate commands to \smartspy.

\begin{lstlisting}[language=Clean,
                   caption={smarts.py run tests example},
                   label=lst:smarts_run_test_example]
>>> python3.py smarts.py -e ./envs/cheyenne.yaml -s ~/mpas_model_changes -t
./mpas_tests run mpas_reg_tests
TEST RESULTS
===============================================
 - mpas_reg_tests - PASSED - "All regression tests passed"
 - compile_test - PASSED - "MPAS compiled succesfully"
 - smoke_test - PASSED - "Smoke test succesfull"
 - bit_for_bit_tests - PASSED - "Bit-for-bit identical"
 - sst_update_test - PASSED - "SST update runs as expected"
\end{lstlisting}


\chapter{Tests}
\label{chap:tests}

The unofficial motto of SMARTS is: "If Python can do it, SMARTS can do it!"
SMARTS' tests have the ability to tests virtually any kind of program or
operation system function, as long as the said program can be ran and checked
for success or failure in Python, it can be a test.

SMARTS tests can contain any number of Python functions, classes, modules or
libraries and any number of external programs or executables.

Python contains a number of different methods and modules for launching and
managing executables and subprocess so launching external executables, such as
a atmospheric model, is possible. 

Each test will at least have a set structure so the SMARTS TestManager can
load, launch and receive the results of tests. Each tests will therefore have
an interface in the form of a single class and a single python function.

This chapter describes the needed interfaces for each test.

\section{Test Structure}
\label{sec:test_structure}

The test interface will consist of a file, a class, and a function which will
reside in their own directory. The directory, file and class will all need to
have the same name. Listing \ref{lst:example_test_structure} shows an example
test directory and two example tests.

\begin{lstlisting}[language=Clean, 
                   caption={Example Test Structure},
                   label={lst:example_test_structure}]
\test_directory
    \test1
        test1.py
    \test2
        test2.py
\end{lstlisting}

The name used for the test directory, test file and the test class are used as
the test launch name.

Each class will need to have the following attributes:

\begin{itemize}
    \item {\tt ncpus} - {\tt int} - Required 
    \item {\tt test\_name} - {\tt str} - Optional
    \item {\tt test\_description} - {\tt str} - Optional
    \item {\tt dependencies} - {\tt list of str} - Optional
\end{itemize}

% Details about the above attributes

The {\tt ncpus} is the only required attribute and will designate the amount of
CPUs that this test will use. Based on this attribute, the TestManager will
use this number to scheduler tests depending on the maximum number allowed
processes.

Each test file will need to define a class of the same name as the test
test directory name and the test file name. The class will need to define a
single function, \run.

The \run function will need to take the following arguments.

\begin{enumerate}
    \item {\tt self} - Reference to this test instance.
    \item {\tt env} - The environment class that contains information on the
    current environment. The parsed env.yaml file can be found in {\tt
    env.env}. See chapters \ref{chap:environment_file_spec} and
    \ref{chap:environment_class} for more information on the environment class.
    \item {\tt result} - The result object which is used to communicated
    results from the test to the {\tt TestManager}.
    \item {\tt src\_dir} - The directory that contains the code to be tested. If
    SMARTS was started from the {\tt smarts.py} command line interface this is
    the directory that was passed via the {\tt -s} command line option. 
    \item {\tt test\_dir} - The path to the test direction. Similar to the {\tt
    src\_dir} argument, if SMARTS was started from the {\tt smarts.py} command
    line interface the {\tt test\_dir} is the directory that was passed via the
    {\tt -t} command line option.
    \item {\tt hpc} - An instance of the HPC class, instantiated with the HPC
    interface queuing system that was specified in the Description section of
    the env.yaml file. This object can be used to schedule jobs upon the
    current HPC (if one is present).
\end{enumerate}

% Details about the above arguments 

\begin{lstlisting}[language=Python, 
                   caption={Example test1.py},
                   label={lst:example_test}]
import os

class test:
    ncpus = 1
    test_name = 'Test 1'
    test_description = 'SMARTS example test'
    dependencies = None

    def run(self, env, result, src_dir, test_dir, hpc):
        if True:
            result.result = "PASSED"
            result.message = "True is true!"
        else:
            result.result = "FAILED"
            result.message = "It appears True is no longer fact"
\end{lstlisting}


\section{Test Results}
\label{sec:results}

Tests are useless if they cannot communicate their results to tester. Each test
is passed an instance of the {\tt Result} class. The {\tt Result} class is
defined in Listing \ref{lst:result_class}. The ResultClass contains attributes
{\tt result} and {\tt msg}

\begin{lstlisting}[language=Python, 
                   caption={Result class definition},
                   label={lst:result_class}]
class ResultClass:
    result = None
    msg = None
    directory = None
\end{lstlisting}

\section{Starting HPC Jobs}
\label{sec:hpc_jobs}

NOTE TO USERS ABOUT SHARED RESOURCES

Tests in SMARTS have the ability to launch HPC jobs from within tests.
Currently, only PBS HPC's are supported, but in the future SLURM will be
supported.

Jobs can be launched via the {\tt HPC} instanced passed into the argument of
every test. Upon reading the specified environment.yaml file on started, SMARTS
will initalize the corosponding HPC class (at this point either PBS or None),
which will be passed to each test.

To have a test determine if its on an HPC machine or not, it can compare the
{\tt HPC} instance to the currently aviable types (currently only {\tt PBS},
but in the future {\tt SLURM}). HPC can currently have the possible values:

\begin{itemize}
    \item {\tt None} - If the HPC is {\tt None}, then the current machine is
    not an HPC.
    \item {\tt "PBS"} - If the HPC is {\tt PBS}, then the machine is a PBS
    machine.
\end{itemize}

Tests can then use the two methods provided to launch a PBS batch job. These
methods are: {\tt HPC.launch\_script} and {\tt HPC.launch\_job}. {\tt
launch\_script} provides a way to launch already created batch scripts while
{\tt launch\_job} provides a method for directly specifying a job.

Both of the HPC job commands above are blocking. So, if a test calls either
{\tt launch\_job} or {\tt launch\_script}, that test will block until the batch
job is completed. Upon successfully completion, \launchscript and \launchjob
will return True; if the job is not able to be successfully submitted to the
queuing system the two functions will return False.

HPC and its two functions, \launchscript and \launchjob provide no
functionality for checking the result of an HPC job, they only provide whether
jobs were successfully submitted to the queue and finished. Just because
\launchscript or \launchjob returns True does not necessary mean the test
completed successfully, only that it was accepted, ran and finished on the HPC.
It is up to the test itself to check the result of the job (i.e. by reading log
or netcdf files etc.).

In order to help facilitate potential problems and to ensure correct job
submission, they HPC logs all commands sent to the HPC and all received
messages from STDIN and STDOUT. If HPC queuing errors occurred, it is best to
check this log file for more information.

The name of the log file will be: {\tt smarts-hpc.SCRIPT-NAME.log}. Where
SCRIPT-NAME is the name of the script used to submit the job (with any
extension).

\subsection{HPC Launch Job}
\label{sec:launchjob}

% HPC Launch test function description
\begin{lstlisting}[language=Python, 
                   caption={HPC.launch\_job},
                   label={lst:hpc.launch_job},
                   float]
def launch_job(self,
               executables,  # List of executables to run
               name,         # Name of the HPC job
               wallTime,     # Walltime in HH:MM:SS
               queue,        # Desired queue
               nNodes,       # Number of nodes
               ncpus,        # Number of CPUS per node
               nMPI,         # Number of MPI tasks per node
               **kwargs):
\end{lstlisting}

\launchjob has the following arguments:

\begin{enumerate}
\item {\tt Executables} - A list of executables to be preformed by the job. For
instance: source an environment file and launch the init\_atmosphere core. These
executables will be ran in the order that they appear. For example: {\tt
executables=['ulimit -s unlimited', 'source ~/setup\_cheyenne', 'mpiexec\_mpt
./init\_atmosphere']}.

\item {\tt name} - The name to give the HPC job (In PBS this is the '-N'
option.)
\item {\tt wallTime} - The wall time in hh:mm:ss
\item {\tt queue} - The desired queue to use.
\item {\tt nNodes} - The number of nodes to use
\item {\tt ncpus} - The number of cpus to use per node
\item {\tt nMPI} - The number of MPI tasks to use per node
\item {\tt**kwargs} - Optional keyword arguments
\begin{itemize} 
\item {\tt shell} - The desire shell to use for the script. This line will
appear at the top of th shell script. The default is: {\tt \#!/usr/bin/env
bash}.
\item {\tt pbs\_options} - *Soon to be just options* - A dictionary of
additional options to use in the script. All key, value pairs of the dictionary
will be added to the script. The key will be the option argument and its
corresponding option will be the value of the option argument. For instance:
{\tt options = \{ '-M' : 'email\_address' \} } will be translated and inserted
to {\tt \#PBS -M email\_address} for a PBS job script.
\item {\tt script\_name} - The desired script name. Default is {\tt
script.pbs}.
\end{itemize}
\end{enumerate}

Given the arguments provided and optional keyword arguments provided to {\tt
HPC.launch\_job}, {\tt HPC.launch\_job} creates a job script (script.pbs) and
calls the corresponding queue submission command on that script.

\subsection{HPC.launch\_script}
\label{sec:hpc_script}

\launchscript allows a test to run a batch script that was created before by
the user. The function definition of \launchscript can be seen in Figure
\ref{lst:hpc.launch_script}. \launchscript takes a single argument, {\tt
script}, which should point to a valid batch script. This script will be
submitted the corresponding HPC workload manager. 

\launchscript has a single optional keyword argument, {\tt cl\_options} which
will be a list of options to pass the command to submit the script. Options and
their arguments (if they have any) should each be separate elements of the
list. For instance: {\tt cl\_options = ['-M', 'email\_address']}. Options are
added to the submission command in the order they appear.

% HPC Launch Script function description
\begin{lstlisting}[language=Python, 
                   caption={HPC.launch\_script},
                   label={lst:hpc.launch_script}]
def launch_script(self, script, **kwargs):
\end{lstlisting}


 \chapter{Environment.yaml Machine Specification}
 \label{chap:environment_file_spec}

Preforming regression tests across multiple machines and with different
compilers and libraries is a necessary part of maintaining quality software.
Especially if that software is intended to be used on a variety of machines
with a variety of different compilers and libraries.

All machines vary in the amount of resources they have present and how they
manage compilers and libraries. A single test itself should not have to worry
about the details of how a specific library is loaded on each specific machine
and instead should be able to solely focus on preforming tests.

The Environment.yaml and the Environment class work in tandem to load
compilers and libraries across different machines to remove this burden from
tests.

Each machine used for testing will have its own unique Environment.yaml file
and are in the form of the machines name followed by the {\tt .yaml} file
extension.

So for instance, the Cheyenne super computer's Environment.yaml file would be
{\tt cheyenne.yaml}, while Casper's would be {\tt casper.yaml}.

The information within an Environment.yaml file includes: the number of CPUs to
use for testing, the type of HPC scheduler that it may use (if any), and the
compilers, libraries, MPI implementations. It also contains information on how
compilers and libraries can be loaded: either by using the LMOD module command
or by setting environment variables.

An Environment.yaml file contains two required section, with one optional
section.

\begin{itemize}
\item Required Sections
    \begin{itemize}
        \item {\tt Description} - Describes general information about the machine
        \item {\tt Modsets} - Describes different compiler, MPI installation and library combinations
    \end{itemize}
\item Optional Sections
    \begin{itemize}
        \item {\tt PBS\_OPTIONS} - Optional default options to be passed to PBS jobs
        \item {\tt SLURM\_OPTIONS} - Optional default options to be passed to Slurm jobs
    \end{itemize}
\end{itemize}

\section{Description}
\label{section:description}

The Description sections describes the machine in general and includes
information to inform SMARTS of the machines name, the number of CPU's
to use, the HPC type (if any), if the machine is to use the LMOD program to
load compilers and libraries.

An example Description section that contains all necessary parts can be found
in Listing \ref{lst:cheyenne_desc_example}.

\begin{lstlisting}[language=yaml, 
                   caption={Examlpe Cheyenne Environment.yaml Description},
                   label={lst:cheyenne_desc_example}]
Description:
    Name: Cheyenne
    Max Cores: 4
    Modules: True
    LMOD_CMD: /glade/u/apps/ch/opt/lmod/8.1.7/lmod/libexec/lmod
    HPC: PBS
\end{lstlisting}

The example yaml file in Listing \ref{lst:cheyenne_desc_example} is the
Description section for the Cheyenne super computer. According to the
description above, SMARTS will know that: it can load libraries, compilers and
MPI implementations via the module command (see specifying libraries below);
that Cheyenne is a super computer and uses the PBS scheduler; and that SMARTS
can use up to four CPUs on the login nodes on Cheyenne.

The {\tt LMOD\_CMD} attribute is the path to the lmod command. On machines that
use lmod, this can be found by printing the value of the {\tt LMOD\_CMD}
environment variable.

NOTE: The {\tt Max Cores} option in the Description specifies the number of
maximum cores that can be used in the location where SMARTS is ran. So, if
SMARTS is ran on the login node of Cheyenne, and {\tt Max Cores} is set to
{\tt 4}, then SMARTS will only use that many CPUs; however, this does not mean
that a test cannot use more than 4 CPUs. A test could, for instance, launch a
job via the HPC's batch node and use more than the specified amount in {\tt Max
Cores}.

\section{HPC Options}
\label{sec:hpc_options}

The \hpcoptions section of the Environment.yaml file specifies default options
that can be passed to an HPC instance. For instance, a user might want to
always have the workload manager send an email on a job completion to their
email, or they may want to always run under a specific account key. If they do,
they can specify those options and arguments in the \hpcoptions.

The \hpcoptions can contain any options that would be use in the job script for
that machine's workload manager.

Tests will need to retrieve the \hpcoptions dictionary from their environment
class instance and pass it to {\tt HPC.launch\_job}. Tests can also edit
existing dictionary items or add new ones by editing the \hpcoptions
dictionary.

\begin{lstlisting}[language=yaml, 
                   caption={Examlpe HPC\_Options},
                   label={lst:hpc_options}]
HPC_OPTIONS:
  M: email@gmail.com 
  q: regular
  j: oe
\end{lstlisting}

\section{Modsets}
\label{sec:modsets}

The Modsets section describes compilers, MPI implementations, libraries and
environment variables. Because Fortran libraries most often need to be built
with the compilers they are built with, modsets are used to describe
combination of a single compiler, an MPI implementation and any number of
libraries and environment variables.

A installation of a compiler, MPI implementation, or library can be described
as either a module or as a environment variable combination. 

Listing \ref{lst:intel_modset_example} contains an example Modset section with
a single modset for a {\tt INTEL-19.0.1} compiler, this specific example is
take from the Cheyenne.yaml environment file.

\begin{lstlisting}[language=yaml, 
                   caption={Example Cheyenne Intel Modset},
                   label={lst:intel_modset_example}]
Modsets:
  ###############
  # INTEL-19.0.2
  ###############
  INTEL-19.0.2:
    Name: intel-19.0.2
    Compiler:
      Name: intel
      Version: 19.0.2
      Module: intel
      Executables:
        - ifort
        - icc
    MPI:
      Module: mpt
      Version: 2.19
      Executables:
        - mpicc
        - mpif90
    Libs:
      - p-netcdf:
        Name: PNETCDF
        Value: /glade/work/duda/libs-intel19.0.2
      - c-netcdf:
        Name: NETCDF
        Value: /glade/work/duda/libs-intel19.0.2
      - pio:
        Name: PIO
        Value: /glade/work/duda/libs-intel19.0.2
      - external_libs:
        Name: MPAS_EXTERNAL_LIBS
        Ealue: "-L${NETCDF}/lib -lhdf5_hl -lhdf5 -ldl -lz"
      - external_includes:
        Name: MPAS_EXTERNAL_INCLUDES
        Value: "-I${NETCDF}/include"
      - JASPERLIB:
        Name: JASPERLIB
        Value: "/glade/u/home/wrfhelp/UNGRIB_LIBRARIES/lib"
      - JASPERINC:
        Name: JASPERINC
        Value: /glade/u/home/wrfhelp/UNGRIB_LIBRARIES/include
      - use_pio2:
        Name: USE_PIO2
        Value: 'true'
      - precision:
        Name: PRECISION
        Value: single
\end{lstlisting}

Modsets can contain three sections: the {\tt Compiler}, {\tt MPI}, and {\tt
Libs} section.  A modset is required to contain the {\tt Compiler} and {\tt
Libs} sections, but the {\tt MPI} section is optional.

\subsection{Compilers}
\label{subsec:modset_compilers}

The compiler section of a modset describes information needed to load a
specific compiler. It includes the compilers: name (\name), version (\version),
a list of compiler executables (\executables) and either the module name
(\module), or the path to the compilers installation directory (\pathname).

While the \name keyword is required in the Compiler section, it is not actively 
used by SMARTS, but provides readability for users and for tests.

The \version keyword serves two purposes. 

First, if \version is specified with the \module keyword, the name specified
with the \module and the version number are appended together to load a
specific version of the compiler. This is equivalent to running {\tt module
load gnu/8.3.0} or {\tt module load gnu/9.1.0}.

Second, the \version keyword is used in conjunction with executables found
under the \executables keyword to confirm the correct compiler has been loaded.
See Chapter \ref{chap:environment_class} for more information on how the
executables section is used when loading a modset.

In the {\tt Intel-19.0.2} modset from Listing \ref{lst:intel_modset_example},
the compiler is specified with the name intel (denoted by the \module keyword),
and, when loaded, will be loaded by using the lmod module Python command to
load intel/19.0.2. This occurs in the same way one would load the {\tt
intel/19.0.2} using the module command via the command line: {\tt module load
intel/19.0.2}. Lastly, when loaded, SMARTs will test to see if the correct
version of {\tt ifort} and {\tt icc}, which are listed in the executables, have
been loaded correctly.

There must only be one compiler section per modset.

\subsection{MPI}
\label{subsec:modset_mpi}

The \mpi section specifies an MPI installation and is specified in the same
manner as the compiler section. Upon being loaded, the executables listed in
\executables will be tested against the version listed in the \version keyword.

The \mpi section is not required to be present in a modset.

\subsection{Libs}
\label{subsec:modset_libs}

The \libs section can contain information on the installation of libraries and
environment variables. The implementation of the \libs is meant to be flexible
and allow tests to access libraries and needed environment variables.

Individual entries to the \libs section can specified as either a module
(Listing \ref{lst:example_library_module}) or an environment variable (Listing
\ref{lst:example_library_env_var}). Libraries can either be specified as with
the \module and optional \version keywords as seen in Listing
\ref{lst:example_library_module} or as a environment variable using the \name
and \valueenv keywords as seen in Listing \ref{lst:example_library_env_var}.

\begin{lstlisting}[language=yaml, 
                   caption={Example Library Module},
                   label={lst:example_library_module}]
- netcdf:
  Module: netcdf
  Version: 3.6.3
\end{lstlisting}

\begin{lstlisting}[language=yaml, 
                   caption={Example Library Environment Variable},
                   label={lst:example_library_env_var}]
- netcdf:
  Name: NETCDF
  Value: /glade/work/duda/libs-intel19.0.2
\end{lstlisting}

The \libs section is required.


\chapter{The Environment Class}
\label{chap:environment_class}

The Environment Class (smarts/env.py) is the interface that tests will use to
load compilers, mpi installations and libraries that are specified in a
Environment.yaml file. 

SMARTS uses the Environment Class internally to load and store a parsed
Envrionment.yaml file, but has public methods that are to be used by tests to
load different Modsets.

Each \testRun method of a test will be passed an instance of the Environment
class that has been loaded with the Environment.yaml that was specified by the
{\tt smarts.py} command line. Each test can use this instance of the
Environment to load modsets as they choose.

The Environment class contains two public routines which tests can use to load
modsets: \listModsets and \loadModset, which are described in the two sections
below.

\section{List Modsets}
\label{sec:list-modsets}

\begin{lstlisting}[language=Python, 
                   caption={list\_modset Definition},
                   label={lst:list_modset_def}]
def list_modsets(self, name=None, *args, **kwargs):
    """ Return a list of modsets found in the parsed environment.yaml file, if
    name is specified then modset names that contain that name will be
    returned.

    Keyword arguments:
    name -- Name to specify specific modset(s) name (String)
    """
\end{lstlisting}

The \listModsets command can be used retrive the names of modsets in a list
which can later be used in the \loadModset command (see Section
\ref{sec:load-modsets}). As seen in Listing \ref{lst:list_modset_def}, if
\listModsets is ran and {\tt name=None} then all the Modsets contained will be
returned; however, if {\tt name} is set, the Modsets that contain the name
specified in name will be retrived.

For instance, if SMARTS was initiated with a Environment.yaml file that
contained the modsets:
{\tt
\begin{itemize}
\item GNU-9.1.0
\item GNU-8.3.0
\item INTEL-19.0.2
\end{itemize}
}

The following calls to {\tt Environment.list\_modset} would return the
following:

{\tt
\begin{itemize}
\item env.list\_modset() - ['GNU-9.1.0', 'GNU-8.3.0', 'INTEL-19.0.2']
\item env.list\_modset(name='GNU-9.1.0') - ['GNU-9.1.0']
\item env.list\_modset(name='GNU') - ['GNU-9.1.0', 'GNU-8.3.0']
\end{itemize}
}


\section{Load Modsets}
\label{sec:load-modsets}

\begin{lstlisting}[language=Python, 
                   caption={load\_modset Definition},
                   label=load_modset_def]
def load_modset(self, modsetName, *args, **kawrgs):
    """ Completely load the modset, modsetName to be used by a single test.
    This function completely loads a modset (compiler, mpi implementation, and
    all libraries).

    To load a compiler, this function will alter the PATH environment variable
    for the current process (single test) and prepend the compiler path to it.
    If the compiler is specified as a module, it will be loaded via the
    lmod Python interface (`module python load ...`)

    MPI implementation will be loaded in a similar manner to compilers.

    Both MPI and Compilers will be checked to ensure that the correct version
    is installed by running the compiler executables specified in the
    executables section of the compiler or MPI env.yaml sections with
    `--version` and checking the versions in the env.yaml file match correctly.

    Libraries will be loaded by creating ENV_NAME as an environment variable
    and assigning to it the value specified in value.

    modsetName -- Name of the modset to be loaded (String)
\end{lstlisting}

From a modset name, returned from \listModsets or entered in manual, can be
used in \loadModset to load a specific compiler i.e.:

\begin{lstlisting}[language=Python]
gnu_modsets = env.list_modsets(name="GNU-9.1.0")
env.load_modset(gnu_modsets[0])
\end{lstlisting}

When \loadModset is called on a modset, it will load the \compiler, \mpi and
\libs section in the following order and in the following way:

\begin{enumerate}

\item Compiler 

Depending on how the compiler is specified in the Environment.yaml file, as a
module or as an installation path, it will be loaded either by using the lmod
{\tt module load} command or by prepending the installation's binary ({\tt
bin}) directory to the proccess \pathenv environment variable. 

The lmod program has the ability to generate Python commands which can be used
to alter the program's environment accordingly if executed. SMARTS will take
these commands generated by LMOD and executed them. See ({\tt
Environment.\_lmod\_load()} for more details.

As stated above, if the compiler is listed as a compiler installation path,
then '{\tt \\bin}' will be appended to the installation path and that path will
be appended to the \pathenv environment variable.

In both cases, loading the compiler in such a way will allow calls to Python's
{\tt os.system()} to use the loaded compiler.

After the above steps, SMARTS will check to see if the correct compiler version
has been loaded. This is done by running all of the executables listed in the
\executables section with {\tt --version} and checking to see if the version
specified in \version is in the output.


\item MPI

If specified, the MPI section will be loaded in the same way as the compiler
section above.

\item Libs 

Libraries in the \libs section of the Environment.yaml file will be loaded in
the order that they are listed. Depending on how the library is specified, as
either a module (\module) or as an environment variable value pair (\name,
\value) the library will be loaded using the lmod command or by creating an
environment variable and setting it equal to value.

As with the compiler and MPI sections above, SMARTS will use the lmod python
tool to generate commands and then execute the commands to update its current
runtime environment. 

For environment variable value pairs, it will create a new environment variable
with \name (or overwrite one if the name is already present) and set it to
\value.

As opposed to how the \compiler and \mpi sections are loaded, libraries are not
tested that they are loaded correctly.

\end{enumerate}


\appendix

\chapter{Example Environment.yaml Files}
\label{chap:environment_file_examples}


\chapter{Example SMARTS Tests}
\label{chap:example_tests}


\end{document}
