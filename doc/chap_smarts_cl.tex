\chapter{Command Line Interface - smarts.py}
\label{chap:smarts_commandline}

SMARTS can currently only be launched from the command line using Python via
\smartspy. In the future there may be other options for launching SMARTS.

The \smartspy command line program is a Python3 program and should be launched
by using Python3. The \smartspy interface is built in the similar manner to the
Git commandline tool in that there are commands and sub-commands. Different to
the Git commandline tool, \smartspy has a number of options that are required
to run.

Listing \ref{lst:smarts_help} shows the usage and help message for \smartspy.
There are three required arguments, which are listed in the 'Required
Arguments' section. These three requirements are:

\begin{itemize}
    \item {\tt -e/--env-file env.yaml} - The Envrionment.yaml file that
    describes the current machine. This file describes the machine and any
    specified compilers or libraries to SMARTS so that tests are able to load
    and unload tests across different machines. See Chapters
    \ref{chap:environment_file_spec} and \ref{chap:environment_class} for more
    information on specifying and using the environment.yaml file.

    \item {\tt -s/--src-dir dir} - The directory of changes to test. This
    directory path will be passed to each test (as the {\tt src\_dir} argument)
    and will allow tests to copy soruce code or executables for regression
    testing. This directory can be located anywhere on the filesystem. See
    Chapter \ref{chap:tests} for more infromation on using the directory from
    the {\tt -s} argument in tests.

    \item {\tt -t/--test-dir dir} - The directory that contains the desired
    tests to run. This argument does two things. One, it shows SMARTS where
    tests can be loaded. SMARTS will look in this directory for valid tests and
    will only load tests from this directory. Secondly, it passes this path to
    each test which allows tests to have access to any supplementary files that
    they may need. For instance this could be a namelist or streams file that
    is specific for a spcific test.
\end{itemize}

All three of these requirments arguments must be specified or \smartspy will
return an error. After these required arguments commands and their subcommands
may be specified. The two commands and their subcommands are:

\begin{itemize}
    \item {\tt list } - The \listcmd can be used to display infromation of the
    current SMARTS system. At present, the list command can only be used to
    list tests and test information, but in the future it may be expanded
    to print other infromation. The subcommands for \listcmd are:
    \begin{itemize}
        \item {\tt list tests} - List the valid and invalid tests that are
        found the the directory passed to the {\tt -t/--test-dir}. See
        \ref{sec:list} for more information.
        \item {\tt list test test-name[s]} - Print out the additional
        infromation on a specific test or test(s). See \ref{sec:list} for
        additional information.
    \end{itemize}
    \item {\tt run test-name[s]} - Run the specified tests. See \ref{sec:run}
    for more information.
\end{itemize}

\begin{lstlisting}[language=Clean,
                   caption={smarts.py Help Message},
                   label=lst:smarts_help]
usage: smarts [-h] [-e env.yaml] [-s dir] [-t dir] [-v level] {list,run} ...

A regression testing system for MPAS

optional arguments:
  -h, --help            show this help message and exit

Required arguments:
  -e env.yaml, --env-file env.yaml
                        The location of the env.yaml file
  -s dir, --src-dir dir
                        The directory that holds the code to test changes
                        (MPAS-Model)
  -t dir, --test-dir dir
                        The location of the test directory

Optional arguments:
  -v level, --verbose level
                        Output debug level

subcommands:
  command description

  {list,run}            Sub-command help message
    list                List SMART's tests, test suites and compilers
    run                 Run a test or a test-suite by name
\end{lstlisting}

\section{List}
\label{sec:list}

An example of the \listtest and its subcommands can be seen in Listing
\ref{lst:smarts_list_tests_example}. The \listtest command will display two
groups of tests. One for valid tests and another for invalid tests. Valid tests
are tests that contain all the necessary parts to be successfully loaded and
launched by SMARTS. Invalid tests, on the other hand, are tests that SMARTS
could not load for one reason or another. Tests that contain syntax errors or
are missing critical parts of a SMARTS test will be placed in invalid tests,
along with the reason they were invalid.

\begin{lstlisting}[language=Clean,
                   caption={smarts.py List Tests Example},
                   label=lst:smarts_list_tests_example]
>>> python3.py smarts.py -e ./envs/cheyenne.yaml -s ~/mpas_model_changes -t
./mpas_tests list tests
Tests found in: /users/mcurry/smarts/smarts_mpas_test:
Valid tests:
  - bit_for_bit_tests -- Bit for Bit Test
  - compile_test -- Test that MPAS compiles
  - mpas_reg_tests -- Complete MPAS Regression Test
  - smoke_test -- Smoke Test
  - sst_update_test -- SST Update Test

Invalid tests: (These tests were not able to be loaded)
  x restart_test.restart_test -- EOL while scanning string literal (restart_test.py, line 12)
\end{lstlisting}

To find out more information on a test, one can run the {\tt list test
[test-name]} command and smarts.py will print out additional information test
information. This command can be used with any number of test names. Listing
\ref{lst:smarts_list_test_name_example} provides an example for this command.

\begin{lstlisting}[language=Clean,
                   caption={smarts.py List Test Info},
                   label={lst:smarts_list_test_name_example}]
>>> python3.py smarts.py -e ./envs/cheyenne.yaml -s ~/mpas_model_changes -t
./mpas_tests list test mpas_reg_tests
    Run name: mpas_reg_tests
   Long name: Complete MPAS Regression Test
 Description: Run all MPAS regression tests
       ncpus: 1
Dependencies: ['compile_test', 'smoke_test', 'bit_for_bit_tests', 
               'sst_update_test']
\end{lstlisting}

\section{Run}
\label{sec:run}

The \runcmd can be used with one or more test names. Doing so, will launch the
specified tests through SMARTS, if those tests are valid. If they are not, an
error will be reported and no tests will be ran.

If all tests are valid and able to be loaded correctly, then SMARTS will create
a new run directory will be created. Within this directory tests will be given
their own directory, which will become their current working directory.
Grabbing the results of one test (say a compiled executable from a compile
test) can be done by using the current working directory path. All main test run
directories will named {\tt run-smarts-YYYY-MM-DD-hh.mm.ss} with the date and
time pieces being the date and time whne SMARTS was launched. 

If a test has a test (or multiple tests) listed in its dependencies attribute,
and that test is not specified in the run command, then it will automatcially
be loaded and ran. Tests that are dependencies for other tests will be ran
before their dependents and dependents will not run if any of their
dependencies fail.

For instance, in \ref{lst:smarts_list_test_name_example} specifying
mpas\_reg\_testing to run will load and run all of the dependencies listed in
the Dependencie's list. If any of these tests fail, then the result of
mpas\_reg\_testing will be marked as {\tt INCOMPLETE}.

Tests will only be loaded once per \smartspy command. It is not possible for
SMARTS to run two of the same tests twice. This includes if a test is specified
as a dependency and is specified to run via \smartspy.

Running the same test twice should be done with seperate commands to \smartspy.

\begin{lstlisting}[language=Clean,
                   caption={smarts.py run tests example},
                   label=lst:smarts_run_test_example]
>>> python3.py smarts.py -e ./envs/cheyenne.yaml -s ~/mpas_model_changes -t
./mpas_tests run mpas_reg_tests
TEST RESULTS
===============================================
 - mpas_reg_tests - PASSED - "All regression tests passed"
 - compile_test - PASSED - "MPAS compiled succesfully"
 - smoke_test - PASSED - "Smoke test succesfull"
 - bit_for_bit_tests - PASSED - "Bit-for-bit identical"
 - sst_update_test - PASSED - "SST update runs as expected"
\end{lstlisting}
