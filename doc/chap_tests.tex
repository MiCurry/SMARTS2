\chapter{Tests}
\label{chap:tests}

The unofficial motto of SMARTS is: "If Python can do it, SMARTS can do it!"
SMARTS' tests have the ability to tests virtually any kind of program or
operation system function, as long as the said program can be ran and checked
for success or failure in Python, it can be a test.

SMARTS tests can contain any number of Python functions, classes, modules or
libraries and any number of external programs or executables.

Python contains a number of different methods and modules for launching and
managing executables and subprocess so launching external executables, such as
a atmospheric model, is possible. 

Each test will at least have a set structure so the SMARTS TestManager can
load, launch and receive the results of tests. Each tests will therefore have
an interface in the form of a single class and a single python function.

This chapter describes the needed interfaces for each test.

\section{Test Structure}
\label{sec:test_structure}

The test interface will consist of a file, a class, and a function which will
reside in their own directory. The directory, file and class will all need to
have the same name. Listing \ref{lst:example_test_structure} shows an example
test directory and two example tests.

\begin{lstlisting}[language=Clean, 
                   caption={Example Test Structure},
                   label={lst:example_test_structure}]
\test_directory
    \test1
        test1.py
    \test2
        test2.py
\end{lstlisting}

The name used for the test directory, test file and the test class are used as
the test launch name.

Each class will need to have the following attributes:

\begin{itemize}
    \item {\tt ncpus} - {\tt int} - Required 
    \item {\tt test\_name} - {\tt str} - Optional
    \item {\tt test\_description} - {\tt str} - Optional
    \item {\tt dependencies} - {\tt list of str} - Optional
\end{itemize}

% Details about the above attributes

The {\tt ncpus} is the only required attribute and will designate the amount of
CPUs that this test will use. Based on this attribute, the TestManager will
use this number to scheduler tests depending on the maximum number allowed
processes.

Each test file will need to define a class of the same name as the test
test directory name and the test file name. The class will need to define a
single function, \run.

The \run function will need to take the following arguments.

\begin{enumerate}
    \item {\tt self} - Reference to this test instance.
    \item {\tt env} - The environment class that contains information on the
    current environment. The parsed env.yaml file can be found in {\tt
    env.env}. See chapters \ref{chap:environment_file_spec} and
    \ref{chap:environment_class} for more information on the environment class.
    \item {\tt result} - The result object which is used to communicated
    results from the test to the {\tt TestManager}.
    \item {\tt src\_dir} - The directory that contains the code to be tested. If
    SMARTS was started from the {\tt smarts.py} command line interface this is
    the directory that was passed via the {\tt -s} command line option. 
    \item {\tt test\_dir} - The path to the test direction. Similar to the {\tt
    src\_dir} argument, if SMARTS was started from the {\tt smarts.py} command
    line interface the {\tt test\_dir} is the directory that was passed via the
    {\tt -t} command line option.
    \item {\tt hpc} - An instance of the HPC class, instantiated with the HPC
    interface queuing system that was specified in the Description section of
    the env.yaml file. This object can be used to schedule jobs upon the
    current HPC (if one is present).
\end{enumerate}

% Details about the above arguments 

\begin{lstlisting}[language=Python, 
                   caption={Example test1.py},
                   label={lst:example_test}]
import os

class test:
    ncpus = 1
    test_name = 'Test 1'
    test_description = 'SMARTS example test'
    dependencies = None

    def run(self, env, result, src_dir, test_dir, hpc):
        if True:
            result.result = "PASSED"
            result.message = "True is true!"
        else:
            result.result = "FAILED"
            result.message = "It appears True is no longer fact"
\end{lstlisting}


\section{Test Results}
\label{sec:results}

Tests are useless if they cannot communicate their results to tester. Each test
is passed an instance of the {\tt Result} class. The {\tt Result} class is
defined in Listing \ref{lst:result_class}. The ResultClass contains attributes
{\tt result} and {\tt msg}

\begin{lstlisting}[language=Python, 
                   caption={Result class definition},
                   label={lst:result_class}]
class ResultClass:
    result = None
    msg = None
    directory = None
\end{lstlisting}

\section{Starting HPC Jobs}
\label{sec:hpc_jobs}

NOTE TO USERS ABOUT SHARED RESOURCES

Tests in SMARTS have the ability to launch HPC jobs from within tests.
Currently, only PBS HPC's are supported, but in the future SLURM will be
supported.

Jobs can be launched via the {\tt HPC} instanced passed into the argument of
every test. Upon reading the specified environment.yaml file on started, SMARTS
will initalize the corosponding HPC class (at this point either PBS or None),
which will be passed to each test.

To have a test determine if its on an HPC machine or not, it can compare the
{\tt HPC} instance to the currently aviable types (currently only {\tt PBS},
but in the future {\tt SLURM}). HPC can currently have the possible values:

\begin{itemize}
    \item {\tt None} - If the HPC is {\tt None}, then the current machine is
    not an HPC.
    \item {\tt "PBS"} - If the HPC is {\tt PBS}, then the machine is a PBS
    machine.
\end{itemize}

Tests can then use the two methods provided to launch a PBS batch job. These
methods are: {\tt HPC.launch\_script} and {\tt HPC.launch\_job}. {\tt
launch\_script} provides a way to launch already created batch scripts while
{\tt launch\_job} provides a method for directly specifying a job.

Both of the HPC job commands above are blocking. So, if a test calls either
{\tt launch\_job} or {\tt launch\_script}, that test will block until the batch
job is completed. Upon successfully completion, \launchscript and \launchjob
will return True; if the job is not able to be successfully submitted to the
queuing system the two functions will return False.

HPC and its two functions, \launchscript and \launchjob provide no
functionality for checking the result of an HPC job, they only provide whether
jobs were successfully submitted to the queue and finished. Just because
\launchscript or \launchjob returns True does not necessary mean the test
completed successfully, only that it was accepted, ran and finished on the HPC.
It is up to the test itself to check the result of the job (i.e. by reading log
or netcdf files etc.).

In order to help facilitate potential problems and to ensure correct job
submission, they HPC logs all commands sent to the HPC and all received
messages from STDIN and STDOUT. If HPC queuing errors occurred, it is best to
check this log file for more information.

The name of the log file will be: {\tt smarts-hpc.SCRIPT-NAME.log}. Where
SCRIPT-NAME is the name of the script used to submit the job (with any
extension).

\subsection{HPC Launch Job}
\label{sec:launchjob}

% HPC Launch test function description
\begin{lstlisting}[language=Python, 
                   caption={HPC.launch\_job},
                   label={lst:hpc.launch_job},
                   float]
def launch_job(self,
               executables,  # List of executables to run
               name,         # Name of the HPC job
               wallTime,     # Walltime in HH:MM:SS
               queue,        # Desired queue
               nNodes,       # Number of nodes
               ncpus,        # Number of CPUS per node
               nMPI,         # Number of MPI tasks per node
               **kwargs):
\end{lstlisting}

\launchjob has the following arguments:

\begin{enumerate}
\item {\tt Executables} - A list of executables to be preformed by the job. For
instance: source an environment file and launch the init\_atmosphere core. These
executables will be ran in the order that they appear. For example: {\tt
executables=['ulimit -s unlimited', 'source ~/setup\_cheyenne', 'mpiexec\_mpt
./init\_atmosphere']}.

\item {\tt name} - The name to give the HPC job (In PBS this is the '-N'
option.)
\item {\tt wallTime} - The wall time in hh:mm:ss
\item {\tt queue} - The desired queue to use.
\item {\tt nNodes} - The number of nodes to use
\item {\tt ncpus} - The number of cpus to use per node
\item {\tt nMPI} - The number of MPI tasks to use per node
\item {\tt**kwargs} - Optional keyword arguments
\begin{itemize} 
\item {\tt shell} - The desire shell to use for the script. This line will
appear at the top of th shell script. The default is: {\tt \#!/usr/bin/env
bash}.
\item {\tt pbs\_options} - *Soon to be just options* - A dictionary of
additional options to use in the script. All key, value pairs of the dictionary
will be added to the script. The key will be the option argument and its
corresponding option will be the value of the option argument. For instance:
{\tt options = \{ '-M' : 'email\_address' \} } will be translated and inserted
to {\tt \#PBS -M email\_address} for a PBS job script.
\item {\tt script\_name} - The desired script name. Default is {\tt
script.pbs}.
\end{itemize}
\end{enumerate}

Given the arguments provided and optional keyword arguments provided to {\tt
HPC.launch\_job}, {\tt HPC.launch\_job} creates a job script (script.pbs) and
calls the corresponding queue submission command on that script.

\subsection{HPC.launch\_script}
\label{sec:hpc_script}

\launchscript allows a test to run a batch script that was created before by
the user. The function definition of \launchscript can be seen in Figure
\ref{lst:hpc.launch_script}. \launchscript takes a single argument, {\tt
script}, which should point to a valid batch script. This script will be
submitted the corresponding HPC workload manager. 

\launchscript has a single optional keyword argument, {\tt cl\_options} which
will be a list of options to pass the command to submit the script. Options and
their arguments (if they have any) should each be separate elements of the
list. For instance: {\tt cl\_options = ['-M', 'email\_address']}. Options are
added to the submission command in the order they appear.

% HPC Launch Script function description
\begin{lstlisting}[language=Python, 
                   caption={HPC.launch\_script},
                   label={lst:hpc.launch_script}]
def launch_script(self, script, **kwargs):
\end{lstlisting}
