\chapter{Tests}
\label{chap:tests}

The unofficial motto of SMARTS is: "If Python can do it, SMARTS can do it!"
SMARTS' tests have the ability to tests virtually any kind of program or
operation system function, as long as the said program can be ran and checked
for success or failure via Python commands, it can be a test.

SMARTS tests can contain any number of Python functions, classes, modules or
libraries and run any number of external programs or executables. 

Python contains a number of different methods and modules for launching and
managing executables and subprocess so launching external executables, such as
an atmospheric model, is possible.

All test will at least have a set structure so the SMARTS TestManager can
load, launch and receive the results of tests. Each tests will therefore have
an interface in the form of a single class and a single python function.

This chapter describes that interfaces.

\section{Test Structure}
\label{sec:test_structure}

All SMARTS complaint tests will consist of (at least) a directory, a file and a
class all with the same name. Listing \ref{lst:example_test_structure} shows an
example test directory and two example tests.

\begin{lstlisting}[language=Clean, 
                   caption={Example Test Structure},
                   label={lst:example_test_structure}]
\test_directory
    \test1
        test1.py
    \test2
        test2.py
\end{lstlisting}

The name used for the test directory, test file and the test class are used as
the test launch name as this is the name that the user will use to run the test
via the \smartspy command line interface.

Each class will need to have the following attributes:

\begin{itemize}
    \item {\tt ncpus} - {\tt int} - Required - The number of CPUs that this
    test will used. When this test is ran, SMARTS will subtract this number
    from the number of available CPUS and will only launch other tests if
    enough CPUs are available.
    \item {\tt test\_name} - {\tt str} - Optional - This is the 'long name' of
    the test and is currently only shown when the \listtest command is run.
    However, providing it offers others who are reading the test (or who have
    run \listtest more details on the test).
    \item {\tt test\_description} - {\tt str} - Optional - Similar to the {\tt
    test\_name} the {\tt test\_description} provides information to other users
    (and perhaps your future self) on the details of specific tests. While it
    is optional it is recommend that a good description of the test be
    provided.
    \item {\tt dependencies} - {\tt list of str} - Optional - If a test is
    dependent upon the result of another test, then that tests launch name
    should be added into the {\tt dependencies} list. If a test list another
    test in its dependencies (or multiple tests) it will only be ran if all of
    its dependents complete successfully. Listing
    \ref{lst:example_test_with_dep} shows an example test with a single
    dependency.
\end{itemize}

Each test file will need to define a class of the same name as the test
test directory name and the test file name. The class will need to define a
single function, \run. The \run function will be starting location of
execution when SMARTS runs a test.

The \run function will need to take the following arguments.

\begin{enumerate}
    \item {\tt self} - Reference to this test instance.
    \item {\tt env} - The environment class that contains information on the
    current environment. See chapters \ref{chap:environment_file_spec} and
    \ref{chap:environment_class} for more information on the environment class.
    \item {\tt result} - The result object which is used to communicated
    results from the test to the {\tt TestManager}.
    \item {\tt src\_dir} - The directory that contains the code to be tested.
    If SMARTS was started from the {\tt smarts.py} command line interface this
    is the directory that was passed via the {\tt -s} command line option. Use
    this directory path to copy or link files into the current working
    directory.
    \item {\tt test\_dir} - The path to the test direction. Similar to the {\tt
    src\_dir} argument, if SMARTS was started from the {\tt smarts.py} command
    line interface the {\tt test\_dir} is the directory that was passed via the
    {\tt -t} command line option. Use this directory path and the name of each
    test to retrieve supplementary test files. For instance, a 
    namelist.atmosphere file used for a idealized simulation.
    \item {\tt hpc} - An instance of the HPC class, instantiated with the HPC
    interface queuing system that was specified in the Description section of
    the env.yaml file. This object can be used to schedule jobs upon the
    current HPC (if one is present). See Section` \ref{sec:hpc_jobs} on starting
    jobs via the HPC class.
\end{enumerate}

\begin{lstlisting}[language=Python, 
                   caption={Example test1.py},
                   label={lst:example_test}]
import os

class test1:
    ncpus = 1
    test_name = 'Test 1'
    test_description = 'SMARTS example test'
    dependencies = None

    def run(self, env, result, src_dir, test_dir, hpc):
        if True:
            result.result = "PASSED"
            result.message = "True is true!"
        else:
            result.result = "FAILED"
            result.message = "It appears True is no longer fact"
\end{lstlisting}

\begin{lstlisting}[language=Python, 
                   caption={Example Test With Dependencies},
                   label={lst:example_test_with_dep}]
import os

class test2:
    ncpus = 2
    test_name = 'Test 2'
    test_description = 'SMARTS example test dependent on Test1'
    dependencies = ['test1']

    def run(self, env, result, src_dir, test_dir, hpc):
        ...
\end{lstlisting}

\section{Test Results}
\label{sec:results}

Tests are useless if they cannot communicate their results to tester. Each test
is passed an instance of the {\tt Result} class. The {\tt Result} class is
defined in Listing \ref{lst:result_class}. The {\tt ResultClass} contains
attributes {\tt result} and {\tt msg}. The {\tt ResultClass} can be see in
Listing \ref{lst:result_class}.

\begin{lstlisting}[language=Python, 
                   caption={Result class definition},
                   label={lst:result_class}]
class ResultClass:
    result = None
    msg = None
    directory = None
\end{lstlisting}

Each test is passed the {\tt ResultClass} object which can be used to
communicate the tests result with SMARTS. Setting {\tt ResultClass.result} to
either {\tt "PASSED} or {\tt "FAILED"} results in a pass or fail from that
specific test respectively. The {\tt ResultClass.msg} attribute is an optional
attribute that can describe the reason for failure (or success) of a test.

These results will be communicated to the user in the tests results report when
all tests finish completion, along with the error message.

After a test completes, SMARTS will unscheduled a test if one of their
dependencies returns a {\tt "FAILED"} result. Unscheduling a test results in
its {\tt ResultClass.result} being set to {\tt "INCOMPLETE"}. If SMARTS updates
a test (based on the {\tt "FAILED"} result from another test it will
recursively unscheduled any test that is dependent upon it.

Setting {\tt Result.result} to {\tt None} has the same effect as setting the
test to {\tt "FAILED"}. {\tt Result.result} is set to {\tt None} by default.

In the future there are plans to have more extensive reporting tools. The main
being a way to place the results of tests and plots into a single LaTeX file
and compiled into a PDF which can enable testers to view tests and plots created
by tests in a single place.

\section{Starting HPC Jobs}
\label{sec:hpc_jobs}

Tests in SMARTS have the ability to launch HPC jobs from within tests.
Currently, only PBS HPC's are supported, but in the future SLURM will also be
supported.

Jobs can be launched via the {\tt HPC} instanced passed into the run function
of every test. Upon reading the specified environment.yaml file on started,
SMARTS will initialize the corresponding HPC class (at this point either PBS or
None), which will be passed to each test.

To have a test determine if its on an HPC machine or not, it can compare the
{\tt HPC} instance to the currently available types (currently only {\tt PBS},
but in the future {\tt SLURM}). HPC can currently have the possible values:

\begin{itemize}
    \item {\tt None} - If the HPC is {\tt None}, then the current machine is
    not an HPC.
    \item {\tt "PBS"} - If the HPC is {\tt PBS}, then the machine is a PBS
    machine.
\end{itemize}

Tests can then use the two methods provided to launch a PBS batch job. These
methods are: {\tt HPC.launch\_script} and {\tt HPC.launch\_job}. {\tt
launch\_script} provides a way to launch already created batch scripts while
{\tt launch\_job} provides a method for directly specifying a job.

Both of the HPC job commands above are blocking. So, if a test calls either
{\tt launch\_job} or {\tt launch\_script}, that test will block until the batch
job is completed. Upon successfully completion, \launchscript and \launchjob
will return True; if the job is not able to be successfully submitted to the
queuing system the two functions will return False.

HPC and its two functions, \launchscript and \launchjob provide no
functionality for checking the result of an HPC job, they only provide whether
jobs were successfully submitted to the queue and finished. Just because
\launchscript or \launchjob returns True does not necessary mean the test
completed successfully, only that it was accepted, ran and finished on the HPC.
It is up to the test itself to check the result of the job (i.e. by reading log
or netcdf files etc.).

In order to help facilitate potential problems and to ensure correct job
submission, they HPC logs all commands sent to the HPC and all received
messages from STDIN and STDOUT. If HPC queuing errors occurred, it is best to
check this log file for more information.

The name of the log file will be: {\tt smarts-hpc.SCRIPT-NAME.log}. Where
SCRIPT-NAME is the name of the script used to submit the job (with any
extension).

\subsection{HPC Launch Job}
\label{sec:launchjob}

% HPC Launch test function description
\begin{lstlisting}[language=Python, 
                   caption={HPC.launch\_job},
                   label={lst:hpc.launch_job},
                   float]
def launch_job(self,
               executables,  # List of executables to run
               name,         # Name of the HPC job
               wallTime,     # Walltime in HH:MM:SS
               queue,        # Desired queue
               nNodes,       # Number of nodes
               ncpus,        # Number of CPUS per node
               nMPI,         # Number of MPI tasks per node
               **kwargs):
\end{lstlisting}

\launchjob has the following arguments:

\begin{enumerate}
\item {\tt Executables} - A list of executables to be preformed by the job. For
instance: source an environment file and launch the init\_atmosphere core. These
executables will be ran in the order that they appear. For example: {\tt
executables=['ulimit -s unlimited', 'source ~/setup\_cheyenne', 'mpiexec\_mpt
./init\_atmosphere']}.

\item {\tt name} - The name to give the HPC job (In PBS this is the '-N'
option.)
\item {\tt wallTime} - The wall time in hh:mm:ss
\item {\tt queue} - The desired queue to use.
\item {\tt nNodes} - The number of nodes to use
\item {\tt ncpus} - The number of cpus to use per node
\item {\tt nMPI} - The number of MPI tasks to use per node
\item {\tt**kwargs} - Optional keyword arguments
\begin{itemize} 
\item {\tt shell} - The desire shell to use for the script. This line will
appear at the top of th shell script. The default is: {\tt \#!/usr/bin/env
bash}.
\item {\tt pbs\_options} - *Soon to be just options* - A dictionary of
additional options to use in the script. All key, value pairs of the dictionary
will be added to the script. The key will be the option argument and its
corresponding option will be the value of the option argument. For instance:
{\tt options = \{ '-M' : 'email\_address' \} } will be translated and inserted
to {\tt \#PBS -M email\_address} for a PBS job script.
\item {\tt script\_name} - The desired script name. Default is {\tt
script.pbs}.
\end{itemize}
\end{enumerate}

Given the arguments provided and optional keyword arguments provided to {\tt
HPC.launch\_job}, {\tt HPC.launch\_job} creates a job script (script.pbs) and
calls the corresponding queue submission command on that script.

\subsection{HPC.launch\_script}
\label{sec:hpc_script}

\launchscript allows a test to run a batch script that was created before by
the user. The function definition of \launchscript can be seen in Figure
\ref{lst:hpc.launch_script}. \launchscript takes a single argument, {\tt
script}, which should point to a valid batch script. This script will be
submitted the corresponding HPC workload manager. 

\launchscript has a single optional keyword argument, {\tt cl\_options} which
will be a list of options to pass the command to submit the script. Options and
their arguments (if they have any) should each be separate elements of the
list. For instance: {\tt cl\_options = ['-M', 'email\_address']}. Options are
added to the submission command in the order they appear.

% HPC Launch Script function description
\begin{lstlisting}[language=Python, 
                   caption={HPC.launch\_script},
                   label={lst:hpc.launch_script}]
def launch_script(self, script, **kwargs):
\end{lstlisting}
